% Options for packages loaded elsewhere
\PassOptionsToPackage{unicode}{hyperref}
\PassOptionsToPackage{hyphens}{url}
%
\documentclass[
]{article}
\usepackage{amsmath,amssymb}
\usepackage{lmodern}
\usepackage{iftex}
\ifPDFTeX
  \usepackage[T1]{fontenc}
  \usepackage[utf8]{inputenc}
  \usepackage{textcomp} % provide euro and other symbols
\else % if luatex or xetex
  \usepackage{unicode-math}
  \defaultfontfeatures{Scale=MatchLowercase}
  \defaultfontfeatures[\rmfamily]{Ligatures=TeX,Scale=1}
\fi
% Use upquote if available, for straight quotes in verbatim environments
\IfFileExists{upquote.sty}{\usepackage{upquote}}{}
\IfFileExists{microtype.sty}{% use microtype if available
  \usepackage[]{microtype}
  \UseMicrotypeSet[protrusion]{basicmath} % disable protrusion for tt fonts
}{}
\makeatletter
\@ifundefined{KOMAClassName}{% if non-KOMA class
  \IfFileExists{parskip.sty}{%
    \usepackage{parskip}
  }{% else
    \setlength{\parindent}{0pt}
    \setlength{\parskip}{6pt plus 2pt minus 1pt}}
}{% if KOMA class
  \KOMAoptions{parskip=half}}
\makeatother
\usepackage{xcolor}
\usepackage[margin=1in]{geometry}
\usepackage{color}
\usepackage{fancyvrb}
\newcommand{\VerbBar}{|}
\newcommand{\VERB}{\Verb[commandchars=\\\{\}]}
\DefineVerbatimEnvironment{Highlighting}{Verbatim}{commandchars=\\\{\}}
% Add ',fontsize=\small' for more characters per line
\usepackage{framed}
\definecolor{shadecolor}{RGB}{48,48,48}
\newenvironment{Shaded}{\begin{snugshade}}{\end{snugshade}}
\newcommand{\AlertTok}[1]{\textcolor[rgb]{1.00,0.81,0.69}{#1}}
\newcommand{\AnnotationTok}[1]{\textcolor[rgb]{0.50,0.62,0.50}{\textbf{#1}}}
\newcommand{\AttributeTok}[1]{\textcolor[rgb]{0.80,0.80,0.80}{#1}}
\newcommand{\BaseNTok}[1]{\textcolor[rgb]{0.86,0.64,0.64}{#1}}
\newcommand{\BuiltInTok}[1]{\textcolor[rgb]{0.80,0.80,0.80}{#1}}
\newcommand{\CharTok}[1]{\textcolor[rgb]{0.86,0.64,0.64}{#1}}
\newcommand{\CommentTok}[1]{\textcolor[rgb]{0.50,0.62,0.50}{#1}}
\newcommand{\CommentVarTok}[1]{\textcolor[rgb]{0.50,0.62,0.50}{\textbf{#1}}}
\newcommand{\ConstantTok}[1]{\textcolor[rgb]{0.86,0.64,0.64}{\textbf{#1}}}
\newcommand{\ControlFlowTok}[1]{\textcolor[rgb]{0.94,0.87,0.69}{#1}}
\newcommand{\DataTypeTok}[1]{\textcolor[rgb]{0.87,0.87,0.75}{#1}}
\newcommand{\DecValTok}[1]{\textcolor[rgb]{0.86,0.86,0.80}{#1}}
\newcommand{\DocumentationTok}[1]{\textcolor[rgb]{0.50,0.62,0.50}{#1}}
\newcommand{\ErrorTok}[1]{\textcolor[rgb]{0.76,0.75,0.62}{#1}}
\newcommand{\ExtensionTok}[1]{\textcolor[rgb]{0.80,0.80,0.80}{#1}}
\newcommand{\FloatTok}[1]{\textcolor[rgb]{0.75,0.75,0.82}{#1}}
\newcommand{\FunctionTok}[1]{\textcolor[rgb]{0.94,0.94,0.56}{#1}}
\newcommand{\ImportTok}[1]{\textcolor[rgb]{0.80,0.80,0.80}{#1}}
\newcommand{\InformationTok}[1]{\textcolor[rgb]{0.50,0.62,0.50}{\textbf{#1}}}
\newcommand{\KeywordTok}[1]{\textcolor[rgb]{0.94,0.87,0.69}{#1}}
\newcommand{\NormalTok}[1]{\textcolor[rgb]{0.80,0.80,0.80}{#1}}
\newcommand{\OperatorTok}[1]{\textcolor[rgb]{0.94,0.94,0.82}{#1}}
\newcommand{\OtherTok}[1]{\textcolor[rgb]{0.94,0.94,0.56}{#1}}
\newcommand{\PreprocessorTok}[1]{\textcolor[rgb]{1.00,0.81,0.69}{\textbf{#1}}}
\newcommand{\RegionMarkerTok}[1]{\textcolor[rgb]{0.80,0.80,0.80}{#1}}
\newcommand{\SpecialCharTok}[1]{\textcolor[rgb]{0.86,0.64,0.64}{#1}}
\newcommand{\SpecialStringTok}[1]{\textcolor[rgb]{0.80,0.58,0.58}{#1}}
\newcommand{\StringTok}[1]{\textcolor[rgb]{0.80,0.58,0.58}{#1}}
\newcommand{\VariableTok}[1]{\textcolor[rgb]{0.80,0.80,0.80}{#1}}
\newcommand{\VerbatimStringTok}[1]{\textcolor[rgb]{0.80,0.58,0.58}{#1}}
\newcommand{\WarningTok}[1]{\textcolor[rgb]{0.50,0.62,0.50}{\textbf{#1}}}
\usepackage{longtable,booktabs,array}
\usepackage{calc} % for calculating minipage widths
% Correct order of tables after \paragraph or \subparagraph
\usepackage{etoolbox}
\makeatletter
\patchcmd\longtable{\par}{\if@noskipsec\mbox{}\fi\par}{}{}
\makeatother
% Allow footnotes in longtable head/foot
\IfFileExists{footnotehyper.sty}{\usepackage{footnotehyper}}{\usepackage{footnote}}
\makesavenoteenv{longtable}
\usepackage{graphicx}
\makeatletter
\def\maxwidth{\ifdim\Gin@nat@width>\linewidth\linewidth\else\Gin@nat@width\fi}
\def\maxheight{\ifdim\Gin@nat@height>\textheight\textheight\else\Gin@nat@height\fi}
\makeatother
% Scale images if necessary, so that they will not overflow the page
% margins by default, and it is still possible to overwrite the defaults
% using explicit options in \includegraphics[width, height, ...]{}
\setkeys{Gin}{width=\maxwidth,height=\maxheight,keepaspectratio}
% Set default figure placement to htbp
\makeatletter
\def\fps@figure{htbp}
\makeatother
\setlength{\emergencystretch}{3em} % prevent overfull lines
\providecommand{\tightlist}{%
  \setlength{\itemsep}{0pt}\setlength{\parskip}{0pt}}
\setcounter{secnumdepth}{-\maxdimen} % remove section numbering
\usepackage{booktabs}
\usepackage{longtable}
\usepackage{array}
\usepackage{multirow}
\usepackage{wrapfig}
\usepackage{float}
\usepackage{colortbl}
\usepackage{pdflscape}
\usepackage{tabu}
\usepackage{threeparttable}
\usepackage{threeparttablex}
\usepackage[normalem]{ulem}
\usepackage{makecell}
\usepackage{xcolor}
\ifLuaTeX
  \usepackage{selnolig}  % disable illegal ligatures
\fi
\IfFileExists{bookmark.sty}{\usepackage{bookmark}}{\usepackage{hyperref}}
\IfFileExists{xurl.sty}{\usepackage{xurl}}{} % add URL line breaks if available
\urlstyle{same} % disable monospaced font for URLs
\hypersetup{
  pdftitle={A2 - Analítica descriptiva e inferencial},
  pdfauthor={Autor: Eduardo Mora González},
  hidelinks,
  pdfcreator={LaTeX via pandoc}}

\title{A2 - Analítica descriptiva e inferencial}
\author{Autor: Eduardo Mora González}
\date{Noviembre 2022}

\begin{document}
\maketitle

{
\setcounter{tocdepth}{2}
\tableofcontents
}
\begin{Shaded}
\begin{Highlighting}[]
 \FunctionTok{library}\NormalTok{(readr)}
\NormalTok{ gpa }\OtherTok{\textless{}{-}} \FunctionTok{read\_csv}\NormalTok{(}\StringTok{"C:/Users/eduar/Dropbox/ESTUDIOS/Estadística avanzada/PEC2/gpa\_clean.csv"}\NormalTok{)}
\end{Highlighting}
\end{Shaded}

\begin{verbatim}
## Rows: 4137 Columns: 11
\end{verbatim}

\begin{verbatim}
## -- Column specification ---------------------------------------------------------------------------------------------------------------------
## Delimiter: ","
## chr (1): gpaletter
## dbl (6): sat, tothrs, hsize, hsrank, hsperc, colgpa
## lgl (4): athlete, female, white, black
\end{verbatim}

\begin{verbatim}
## 
## i Use `spec()` to retrieve the full column specification for this data.
## i Specify the column types or set `show_col_types = FALSE` to quiet this message.
\end{verbatim}

\begin{Shaded}
\begin{Highlighting}[]
\CommentTok{\# Numero de campos, registros y tipos las variables}
\FunctionTok{str}\NormalTok{(gpa)}
\end{Highlighting}
\end{Shaded}

\begin{verbatim}
## spec_tbl_df [4,137 x 11] (S3: spec_tbl_df/tbl_df/tbl/data.frame)
##  $ sat      : num [1:4137] 920 1170 810 940 1180 980 880 980 1240 1230 ...
##  $ tothrs   : num [1:4137] 43 18 14 40 18 114 78 55 18 17 ...
##  $ hsize    : num [1:4137] 0.1 9.4 1.19 5.71 2.14 ...
##  $ hsrank   : num [1:4137] 4 191 42 252 86 41 161 101 161 3 ...
##  $ hsperc   : num [1:4137] 40 20.3 35.3 44.1 40.2 ...
##  $ colgpa   : num [1:4137] 2.04 4 1.78 2.42 2.61 ...
##  $ athlete  : logi [1:4137] TRUE FALSE TRUE FALSE FALSE FALSE ...
##  $ female   : logi [1:4137] TRUE FALSE FALSE FALSE FALSE TRUE ...
##  $ white    : logi [1:4137] FALSE TRUE TRUE TRUE TRUE TRUE ...
##  $ black    : logi [1:4137] FALSE FALSE FALSE FALSE FALSE FALSE ...
##  $ gpaletter: chr [1:4137] "C" "A" "C" "C" ...
##  - attr(*, "spec")=
##   .. cols(
##   ..   sat = col_double(),
##   ..   tothrs = col_double(),
##   ..   hsize = col_double(),
##   ..   hsrank = col_double(),
##   ..   hsperc = col_double(),
##   ..   colgpa = col_double(),
##   ..   athlete = col_logical(),
##   ..   female = col_logical(),
##   ..   white = col_logical(),
##   ..   black = col_logical(),
##   ..   gpaletter = col_character()
##   .. )
##  - attr(*, "problems")=<externalptr>
\end{verbatim}

\hypertarget{estaduxedstica-descriptiva-y-visualizaciuxf3n}{%
\section{Estadística descriptiva y
visualización}\label{estaduxedstica-descriptiva-y-visualizaciuxf3n}}

\hypertarget{anuxe1lisis-descriptivo}{%
\subsection{Análisis descriptivo}\label{anuxe1lisis-descriptivo}}

\begin{Shaded}
\begin{Highlighting}[]
\CommentTok{\# Análisis descriptivo básico de las variables}
\FunctionTok{summary}\NormalTok{(gpa)}
\end{Highlighting}
\end{Shaded}

\begin{verbatim}
##       sat           tothrs           hsize          hsrank           hsperc            colgpa       athlete          female       
##  Min.   : 470   Min.   :  6.00   Min.   :0.03   Min.   :  1.00   Min.   : 0.1667   Min.   :0.000   Mode :logical   Mode :logical  
##  1st Qu.: 940   1st Qu.: 17.00   1st Qu.:1.65   1st Qu.: 11.00   1st Qu.: 6.4328   1st Qu.:2.210   FALSE:3943      FALSE:2277     
##  Median :1030   Median : 47.00   Median :2.51   Median : 30.00   Median :14.5833   Median :2.660   TRUE :194       TRUE :1860     
##  Mean   :1030   Mean   : 52.83   Mean   :2.80   Mean   : 52.83   Mean   :19.2371   Mean   :2.654                                  
##  3rd Qu.:1120   3rd Qu.: 80.00   3rd Qu.:3.68   3rd Qu.: 70.00   3rd Qu.:27.7108   3rd Qu.:3.120                                  
##  Max.   :1540   Max.   :137.00   Max.   :9.40   Max.   :634.00   Max.   :92.0000   Max.   :4.000                                  
##    white           black          gpaletter        
##  Mode :logical   Mode :logical   Length:4137       
##  FALSE:308       FALSE:3908      Class :character  
##  TRUE :3829      TRUE :229       Mode  :character  
##                                                    
##                                                    
## 
\end{verbatim}

\begin{Shaded}
\begin{Highlighting}[]
\CommentTok{\# Dimensiones}
\FunctionTok{dim}\NormalTok{(gpa)}
\end{Highlighting}
\end{Shaded}

\begin{verbatim}
## [1] 4137   11
\end{verbatim}

\hypertarget{visualizaciuxf3n}{%
\subsection{Visualización}\label{visualizaciuxf3n}}

\hypertarget{distribuciuxf3n-de-las-variables-sat-y-colgpa}{%
\subsubsection{Distribución de las variables 'sat` y
'colgpa`}\label{distribuciuxf3n-de-las-variables-sat-y-colgpa}}

\begin{Shaded}
\begin{Highlighting}[]
\FunctionTok{hist}\NormalTok{(gpa}\SpecialCharTok{$}\NormalTok{sat,}\AttributeTok{col=}\StringTok{"lightsalmon"}\NormalTok{,}\AttributeTok{main=}\StringTok{"Histograma Variable SAT"}\NormalTok{)}
\end{Highlighting}
\end{Shaded}

\includegraphics{edumoraglez-PEC2_files/figure-latex/unnamed-chunk-4-1.pdf}

\begin{Shaded}
\begin{Highlighting}[]
\FunctionTok{boxplot}\NormalTok{(gpa}\SpecialCharTok{$}\NormalTok{sat,}\AttributeTok{col=}\StringTok{"lightsalmon"}\NormalTok{,}\AttributeTok{main=}\StringTok{"Diagrama de Caja Variable SAT"}\NormalTok{)}
\end{Highlighting}
\end{Shaded}

\includegraphics{edumoraglez-PEC2_files/figure-latex/unnamed-chunk-4-2.pdf}

\begin{Shaded}
\begin{Highlighting}[]
\FunctionTok{boxplot}\NormalTok{(gpa}\SpecialCharTok{$}\NormalTok{sat,}\AttributeTok{col=}\StringTok{"lightsalmon"}\NormalTok{,}\AttributeTok{main=}\StringTok{"Diagrama de Caja Variable SAT con los valores"}\NormalTok{)}

\CommentTok{\# Points}
\FunctionTok{stripchart}\NormalTok{(gpa}\SpecialCharTok{$}\NormalTok{sat,             }
           \AttributeTok{method =} \StringTok{"jitter"}\NormalTok{, }
           \AttributeTok{pch =} \DecValTok{1}\NormalTok{,         }
           \AttributeTok{col =} \DecValTok{4}\NormalTok{,          }
           \AttributeTok{vertical =} \ConstantTok{TRUE}\NormalTok{,  }
           \AttributeTok{add =} \ConstantTok{TRUE}\NormalTok{)    }
\end{Highlighting}
\end{Shaded}

\includegraphics{edumoraglez-PEC2_files/figure-latex/unnamed-chunk-4-3.pdf}

\begin{Shaded}
\begin{Highlighting}[]
\FunctionTok{hist}\NormalTok{(gpa}\SpecialCharTok{$}\NormalTok{colgpa,}\AttributeTok{col=}\StringTok{"lightsalmon"}\NormalTok{,}\AttributeTok{main=}\StringTok{"Histograma Variable colgpa"}\NormalTok{)}
\end{Highlighting}
\end{Shaded}

\includegraphics{edumoraglez-PEC2_files/figure-latex/unnamed-chunk-5-1.pdf}

\begin{Shaded}
\begin{Highlighting}[]
\FunctionTok{boxplot}\NormalTok{(gpa}\SpecialCharTok{$}\NormalTok{colgpa,}\AttributeTok{col=}\StringTok{"lightsalmon"}\NormalTok{,}\AttributeTok{main=}\StringTok{"Diagrama de Caja Variable colgpa"}\NormalTok{)}
\end{Highlighting}
\end{Shaded}

\includegraphics{edumoraglez-PEC2_files/figure-latex/unnamed-chunk-5-2.pdf}

\begin{Shaded}
\begin{Highlighting}[]
\FunctionTok{boxplot}\NormalTok{(gpa}\SpecialCharTok{$}\NormalTok{colgpa,}\AttributeTok{col=}\StringTok{"lightsalmon"}\NormalTok{,}\AttributeTok{main=}\StringTok{"Diagrama de Caja Variable colgpa con los valores"}\NormalTok{)}

\CommentTok{\# Points}
\FunctionTok{stripchart}\NormalTok{(gpa}\SpecialCharTok{$}\NormalTok{colgpa,             }
           \AttributeTok{method =} \StringTok{"jitter"}\NormalTok{, }
           \AttributeTok{pch =} \DecValTok{1}\NormalTok{,         }
           \AttributeTok{col =} \DecValTok{4}\NormalTok{,          }
           \AttributeTok{vertical =} \ConstantTok{TRUE}\NormalTok{,  }
           \AttributeTok{add =} \ConstantTok{TRUE}\NormalTok{)    }
\end{Highlighting}
\end{Shaded}

\includegraphics{edumoraglez-PEC2_files/figure-latex/unnamed-chunk-5-3.pdf}

\hypertarget{distribuciuxf3n-de-la-variable-sat-con-respecto-a-la-variable-guxe9nero-female-la-variable-atleta-athlete-y-la-raza-white-black}{%
\subsubsection{Distribución de la variable 'sat` con respecto a la
variable género ('female`), la variable atleta ('athlete`) y la raza
('white`,
'black`)}\label{distribuciuxf3n-de-la-variable-sat-con-respecto-a-la-variable-guxe9nero-female-la-variable-atleta-athlete-y-la-raza-white-black}}

\begin{Shaded}
\begin{Highlighting}[]
\NormalTok{Sum1 }\OtherTok{\textless{}{-}} \FunctionTok{summarize}\NormalTok{( }\FunctionTok{group\_by}\NormalTok{(gpa, female), }\AttributeTok{n=}\FunctionTok{length}\NormalTok{(female), }\AttributeTok{sat=}\FunctionTok{mean}\NormalTok{(sat))}
\NormalTok{Sum2 }\OtherTok{\textless{}{-}} \FunctionTok{summarize}\NormalTok{( }\FunctionTok{group\_by}\NormalTok{(gpa, athlete), }\AttributeTok{n=}\FunctionTok{length}\NormalTok{(athlete), }\AttributeTok{sat=}\FunctionTok{mean}\NormalTok{(sat))}
\NormalTok{Sum3 }\OtherTok{\textless{}{-}} \FunctionTok{summarize}\NormalTok{( }\FunctionTok{group\_by}\NormalTok{(gpa, white), }\AttributeTok{n=}\FunctionTok{length}\NormalTok{(white), }\AttributeTok{sat=}\FunctionTok{mean}\NormalTok{(sat))}
\NormalTok{Sum4 }\OtherTok{\textless{}{-}} \FunctionTok{summarize}\NormalTok{( }\FunctionTok{group\_by}\NormalTok{(gpa, black), }\AttributeTok{n=}\FunctionTok{length}\NormalTok{(black), }\AttributeTok{sat=}\FunctionTok{mean}\NormalTok{(sat))}
\NormalTok{g1 }\OtherTok{\textless{}{-}}  \FunctionTok{ggplot}\NormalTok{( Sum1, }\FunctionTok{aes}\NormalTok{(}\AttributeTok{x=}\NormalTok{female, }\AttributeTok{y=}\NormalTok{sat, }\AttributeTok{fill=}\NormalTok{female)) }\SpecialCharTok{+} \FunctionTok{geom\_bar}\NormalTok{(}\AttributeTok{width =} \DecValTok{1}\NormalTok{, }\AttributeTok{stat =} \StringTok{"identity"}\NormalTok{)}
\NormalTok{g2 }\OtherTok{\textless{}{-}}  \FunctionTok{ggplot}\NormalTok{( Sum2, }\FunctionTok{aes}\NormalTok{(}\AttributeTok{x=}\NormalTok{athlete, }\AttributeTok{y=}\NormalTok{sat, }\AttributeTok{fill=}\NormalTok{athlete)) }\SpecialCharTok{+} \FunctionTok{geom\_bar}\NormalTok{(}\AttributeTok{width =} \DecValTok{1}\NormalTok{, }\AttributeTok{stat =} \StringTok{"identity"}\NormalTok{)}
\NormalTok{g3 }\OtherTok{\textless{}{-}}  \FunctionTok{ggplot}\NormalTok{( Sum3, }\FunctionTok{aes}\NormalTok{(}\AttributeTok{x=}\NormalTok{white, }\AttributeTok{y=}\NormalTok{sat, }\AttributeTok{fill=}\NormalTok{white)) }\SpecialCharTok{+} \FunctionTok{geom\_bar}\NormalTok{(}\AttributeTok{width =} \DecValTok{1}\NormalTok{, }\AttributeTok{stat =} \StringTok{"identity"}\NormalTok{)}
\NormalTok{g4 }\OtherTok{\textless{}{-}}  \FunctionTok{ggplot}\NormalTok{( Sum4, }\FunctionTok{aes}\NormalTok{(}\AttributeTok{x=}\NormalTok{black, }\AttributeTok{y=}\NormalTok{sat, }\AttributeTok{fill=}\NormalTok{black)) }\SpecialCharTok{+} \FunctionTok{geom\_bar}\NormalTok{(}\AttributeTok{width =} \DecValTok{1}\NormalTok{, }\AttributeTok{stat =} \StringTok{"identity"}\NormalTok{)}

\FunctionTok{grid.arrange}\NormalTok{(g1,g2, g3, g4, }\AttributeTok{nrow=}\DecValTok{2}\NormalTok{)}
\end{Highlighting}
\end{Shaded}

\includegraphics{edumoraglez-PEC2_files/figure-latex/unnamed-chunk-6-1.pdf}

\hypertarget{realizad-el-mismo-tipo-de-visualizaciones-con-la-variable-colgpa-y-las-variables-female-athlete-y-whiteblack.}{%
\subsubsection{Realizad el mismo tipo de visualizaciones con la variable
'colgpa` y las variables 'female`, 'athlete` y
'white`/'black`.}\label{realizad-el-mismo-tipo-de-visualizaciones-con-la-variable-colgpa-y-las-variables-female-athlete-y-whiteblack.}}

\begin{Shaded}
\begin{Highlighting}[]
\NormalTok{Sum1 }\OtherTok{\textless{}{-}} \FunctionTok{summarize}\NormalTok{( }\FunctionTok{group\_by}\NormalTok{(gpa, female), }\AttributeTok{n=}\FunctionTok{length}\NormalTok{(female), }\AttributeTok{colgpa=}\FunctionTok{mean}\NormalTok{(colgpa))}
\NormalTok{Sum2 }\OtherTok{\textless{}{-}} \FunctionTok{summarize}\NormalTok{( }\FunctionTok{group\_by}\NormalTok{(gpa, athlete), }\AttributeTok{n=}\FunctionTok{length}\NormalTok{(athlete), }\AttributeTok{colgpa=}\FunctionTok{mean}\NormalTok{(colgpa))}
\NormalTok{Sum3 }\OtherTok{\textless{}{-}} \FunctionTok{summarize}\NormalTok{( }\FunctionTok{group\_by}\NormalTok{(gpa, white), }\AttributeTok{n=}\FunctionTok{length}\NormalTok{(white), }\AttributeTok{colgpa=}\FunctionTok{mean}\NormalTok{(colgpa))}
\NormalTok{Sum4 }\OtherTok{\textless{}{-}} \FunctionTok{summarize}\NormalTok{( }\FunctionTok{group\_by}\NormalTok{(gpa, black), }\AttributeTok{n=}\FunctionTok{length}\NormalTok{(black), }\AttributeTok{colgpa=}\FunctionTok{mean}\NormalTok{(colgpa))}
\NormalTok{g1 }\OtherTok{\textless{}{-}}  \FunctionTok{ggplot}\NormalTok{( Sum1, }\FunctionTok{aes}\NormalTok{(}\AttributeTok{x=}\NormalTok{female, }\AttributeTok{y=}\NormalTok{colgpa, }\AttributeTok{fill=}\NormalTok{female)) }\SpecialCharTok{+} \FunctionTok{geom\_bar}\NormalTok{(}\AttributeTok{width =} \DecValTok{1}\NormalTok{, }\AttributeTok{stat =} \StringTok{"identity"}\NormalTok{)}
\NormalTok{g2 }\OtherTok{\textless{}{-}}  \FunctionTok{ggplot}\NormalTok{( Sum2, }\FunctionTok{aes}\NormalTok{(}\AttributeTok{x=}\NormalTok{athlete, }\AttributeTok{y=}\NormalTok{colgpa, }\AttributeTok{fill=}\NormalTok{athlete)) }\SpecialCharTok{+} \FunctionTok{geom\_bar}\NormalTok{(}\AttributeTok{width =} \DecValTok{1}\NormalTok{, }\AttributeTok{stat =} \StringTok{"identity"}\NormalTok{)}
\NormalTok{g3 }\OtherTok{\textless{}{-}}  \FunctionTok{ggplot}\NormalTok{( Sum3, }\FunctionTok{aes}\NormalTok{(}\AttributeTok{x=}\NormalTok{white, }\AttributeTok{y=}\NormalTok{colgpa, }\AttributeTok{fill=}\NormalTok{white)) }\SpecialCharTok{+} \FunctionTok{geom\_bar}\NormalTok{(}\AttributeTok{width =} \DecValTok{1}\NormalTok{, }\AttributeTok{stat =} \StringTok{"identity"}\NormalTok{)}
\NormalTok{g4 }\OtherTok{\textless{}{-}}  \FunctionTok{ggplot}\NormalTok{( Sum4, }\FunctionTok{aes}\NormalTok{(}\AttributeTok{x=}\NormalTok{black, }\AttributeTok{y=}\NormalTok{colgpa, }\AttributeTok{fill=}\NormalTok{black)) }\SpecialCharTok{+} \FunctionTok{geom\_bar}\NormalTok{(}\AttributeTok{width =} \DecValTok{1}\NormalTok{, }\AttributeTok{stat =} \StringTok{"identity"}\NormalTok{)}

\FunctionTok{grid.arrange}\NormalTok{(g1,g2, g3, g4, }\AttributeTok{nrow=}\DecValTok{2}\NormalTok{)}
\end{Highlighting}
\end{Shaded}

\includegraphics{edumoraglez-PEC2_files/figure-latex/unnamed-chunk-7-1.pdf}

\hypertarget{interpretad-los-gruxe1ficos-brevemente}{%
\subsubsection{Interpretad los gráficos
brevemente}\label{interpretad-los-gruxe1ficos-brevemente}}

La variable SAT, como se puede ver en el histograma y en el diagrama de
caja tiene una distribución normal. Cuando se mete todos los datos como
puntos dentro del diagrama de caja se ve que están todo sobre la media.

Por otro lado, la variable COLPA, la mayor parte de los datos se
encuentran en la mitad superior, como se puede comprobar en el
histograma y diagrama de caja, además cuando se mete todos los datos
como puntos, se confirma esta distribución de los datos.

Cuando comparamos la variable SAT con otras, nos damos cuenta de que la
nota de acceso (SAT) es mayor para los hombres que para las mujeres,
aunque la diferencia no es muy significativa. Al contrario del sexo, los
no deportistas tienen mejor nota de acceso que los que si lo son. Si lo
comparamos con la raza, los blancos tienen una mayor nota que los que
son de raza negra.

Si comparamos la nota media (Colpa), nos damos cuenta de que las mujeres
tienen mejor nota que los hombres. Ocurre igual para las personas que no
son deportistas, tienen mejor nota media. Respecto a la raza los blancos
tienen mejor nota media que los de raza negra.

\hypertarget{intervalo-de-confianza-de-la-media-poblacional-de-la-variable-sat-y-colgpa}{%
\section{Intervalo de confianza de la media poblacional de la variable
sat y
colgpa}\label{intervalo-de-confianza-de-la-media-poblacional-de-la-variable-sat-y-colgpa}}

\hypertarget{supuestos}{%
\subsection{Supuestos}\label{supuestos}}

Los supuestos para le intervalo de confianza es si se puede suponer una
distribución normal y como se ha visto en el apartado anterior, ambas
variables la tienen.

\hypertarget{funciuxf3n-de-cuxe1lculo-del-intervalo-de-confianza}{%
\subsection{Función de cálculo del intervalo de
confianza}\label{funciuxf3n-de-cuxe1lculo-del-intervalo-de-confianza}}

\begin{Shaded}
\begin{Highlighting}[]
\NormalTok{IC }\OtherTok{\textless{}{-}} \ControlFlowTok{function}\NormalTok{(x, NC)\{}
\NormalTok{  n }\OtherTok{\textless{}{-}} \FunctionTok{length}\NormalTok{(x)}
\NormalTok{  errorTipic }\OtherTok{\textless{}{-}} \FunctionTok{sd}\NormalTok{(x) }\SpecialCharTok{/} \FunctionTok{sqrt}\NormalTok{( n )}
\NormalTok{  errorTipic}
\NormalTok{  t}\OtherTok{\textless{}{-}}\FunctionTok{qnorm}\NormalTok{( }\DecValTok{1}\SpecialCharTok{{-}}\NormalTok{NC}\SpecialCharTok{/}\DecValTok{2}\NormalTok{ )}
\NormalTok{  t}
\NormalTok{  error}\OtherTok{\textless{}{-}}\NormalTok{ t }\SpecialCharTok{*}\NormalTok{ errorTipic}
\NormalTok{  error}
  \FunctionTok{return}\NormalTok{ ( }\FunctionTok{c}\NormalTok{( }\FunctionTok{mean}\NormalTok{(x) }\SpecialCharTok{{-}}\NormalTok{ error, }\FunctionTok{mean}\NormalTok{(x) }\SpecialCharTok{+}\NormalTok{ error ))}
\NormalTok{\}}
\end{Highlighting}
\end{Shaded}

\hypertarget{intervalo-de-confianza-de-la-variable-sat}{%
\subsection{Intervalo de confianza de la variable
sat}\label{intervalo-de-confianza-de-la-variable-sat}}

\hypertarget{intervalo-confianza-al-90}\label{intervalo-confianza-al-90}}

\begin{Shaded}
\begin{Highlighting}[]
\CommentTok{\#Intervalo Confianza}

\NormalTok{ic\_90 }\OtherTok{\textless{}{-}} \FunctionTok{IC}\NormalTok{( gpa}\SpecialCharTok{$}\NormalTok{sat, }\AttributeTok{NC=}\FloatTok{0.1}\NormalTok{ ); ic\_90}
\end{Highlighting}
\end{Shaded}

\begin{verbatim}
## [1] 1026.766 1033.896
\end{verbatim}

\begin{Shaded}
\begin{Highlighting}[]
\CommentTok{\#Comprobación}
\FunctionTok{z.test}\NormalTok{(gpa}\SpecialCharTok{$}\NormalTok{sat, }\AttributeTok{sigma.x=}\FunctionTok{sd}\NormalTok{(gpa}\SpecialCharTok{$}\NormalTok{sat), }\AttributeTok{conf.level =} \FloatTok{0.9}\NormalTok{)}
\end{Highlighting}
\end{Shaded}

\begin{verbatim}
## 
##  One-sample z-Test
## 
## data:  gpa$sat
## z = 475.39, p-value < 2.2e-16
## alternative hypothesis: true mean is not equal to 0
## 90 percent confidence interval:
##  1026.766 1033.896
## sample estimates:
## mean of x 
##  1030.331
\end{verbatim}

\hypertarget{intervalo-confianza-al-95}\label{intervalo-confianza-al-95}}

\begin{Shaded}
\begin{Highlighting}[]
\CommentTok{\#Intervalo Confianza}

\NormalTok{ic\_95 }\OtherTok{\textless{}{-}} \FunctionTok{IC}\NormalTok{( gpa}\SpecialCharTok{$}\NormalTok{sat, }\AttributeTok{NC=}\FloatTok{0.05}\NormalTok{ ); ic\_95}
\end{Highlighting}
\end{Shaded}

\begin{verbatim}
## [1] 1026.083 1034.579
\end{verbatim}

\begin{Shaded}
\begin{Highlighting}[]
\CommentTok{\#Comprobación}
\FunctionTok{z.test}\NormalTok{(gpa}\SpecialCharTok{$}\NormalTok{sat, }\AttributeTok{sigma.x=}\FunctionTok{sd}\NormalTok{(gpa}\SpecialCharTok{$}\NormalTok{sat), }\AttributeTok{conf.level =} \FloatTok{0.95}\NormalTok{)}
\end{Highlighting}
\end{Shaded}

\begin{verbatim}
## 
##  One-sample z-Test
## 
## data:  gpa$sat
## z = 475.39, p-value < 2.2e-16
## alternative hypothesis: true mean is not equal to 0
## 95 percent confidence interval:
##  1026.083 1034.579
## sample estimates:
## mean of x 
##  1030.331
\end{verbatim}

\hypertarget{intervalo-de-confianza-de-la-variable-colgpa}{%
\subsection{Intervalo de confianza de la variable
colgpa}\label{intervalo-de-confianza-de-la-variable-colgpa}}

\hypertarget{intervalo-confianza-al-90-1}\label{intervalo-confianza-al-90-1}}

\begin{Shaded}
\begin{Highlighting}[]
\CommentTok{\#Intervalo Confianza}

\NormalTok{ic\_90 }\OtherTok{\textless{}{-}} \FunctionTok{IC}\NormalTok{( gpa}\SpecialCharTok{$}\NormalTok{colgpa, }\AttributeTok{NC=}\FloatTok{0.1}\NormalTok{ ); ic\_90}
\end{Highlighting}
\end{Shaded}

\begin{verbatim}
## [1] 2.637309 2.670953
\end{verbatim}

\begin{Shaded}
\begin{Highlighting}[]
\CommentTok{\#Comprobación}
\FunctionTok{z.test}\NormalTok{(gpa}\SpecialCharTok{$}\NormalTok{colgpa, }\AttributeTok{sigma.x=}\FunctionTok{sd}\NormalTok{(gpa}\SpecialCharTok{$}\NormalTok{colgpa), }\AttributeTok{conf.level =} \FloatTok{0.9}\NormalTok{)}
\end{Highlighting}
\end{Shaded}

\begin{verbatim}
## 
##  One-sample z-Test
## 
## data:  gpa$colgpa
## z = 259.52, p-value < 2.2e-16
## alternative hypothesis: true mean is not equal to 0
## 90 percent confidence interval:
##  2.637309 2.670953
## sample estimates:
## mean of x 
##  2.654131
\end{verbatim}

\hypertarget{intervalo-confianza-al-95-1}\label{intervalo-confianza-al-95-1}}

\begin{Shaded}
\begin{Highlighting}[]
\CommentTok{\#Intervalo Confianza}

\NormalTok{ic\_95 }\OtherTok{\textless{}{-}} \FunctionTok{IC}\NormalTok{( gpa}\SpecialCharTok{$}\NormalTok{colgpa, }\AttributeTok{NC=}\FloatTok{0.05}\NormalTok{ ); ic\_95}
\end{Highlighting}
\end{Shaded}

\begin{verbatim}
## [1] 2.634086 2.674176
\end{verbatim}

\begin{Shaded}
\begin{Highlighting}[]
\CommentTok{\#Comprobación}
\FunctionTok{z.test}\NormalTok{(gpa}\SpecialCharTok{$}\NormalTok{colgpa, }\AttributeTok{sigma.x=}\FunctionTok{sd}\NormalTok{(gpa}\SpecialCharTok{$}\NormalTok{colgpa), }\AttributeTok{conf.level =} \FloatTok{0.95}\NormalTok{)}
\end{Highlighting}
\end{Shaded}

\begin{verbatim}
## 
##  One-sample z-Test
## 
## data:  gpa$colgpa
## z = 259.52, p-value < 2.2e-16
## alternative hypothesis: true mean is not equal to 0
## 95 percent confidence interval:
##  2.634086 2.674176
## sample estimates:
## mean of x 
##  2.654131
\end{verbatim}

\hypertarget{interpretaciuxf3n}{%
\subsection{Interpretación}\label{interpretaciuxf3n}}

El intervalo de confianza del 90\% de la media poblacional de Sat es
(1026.766, 1033.896), esto quiere decir que, si se sacan diferentes
muestras de la población, el 90\% de los intervalos calculados contienen
el valor de la media poblacional.

El intervalo de confianza del 95\% de la media poblacional de Sat es
(1026.083, 1034.579), esto quiere decir que, si se sacan diferentes
muestras de la población, el 95\% de los intervalos calculados contienen
el valor de la media poblacional.

El intervalo de confianza del 90\% de la media poblacional de Colgpa es
(2.637309, 2.670953), esto quiere decir que, si se sacan diferentes
muestras de la población, el 90\% de los intervalos calculados contienen
el valor de la media poblacional.

El intervalo de confianza del 95\% de la media poblacional de Colgpa es
(2.634086, 2.674176), esto quiere decir que, si se sacan diferentes
muestras de la población, el 95\% de los intervalos calculados contienen
el valor de la media poblacional.

\hypertarget{ser-atleta-influye-en-la-nota}{%
\section{¿Ser atleta influye en la
nota?}\label{ser-atleta-influye-en-la-nota}}

\hypertarget{anuxe1lisis-visual}{%
\subsection{Análisis visual}\label{anuxe1lisis-visual}}

\begin{Shaded}
\begin{Highlighting}[]
\NormalTok{V\_athlete }\OtherTok{\textless{}{-}} \FunctionTok{filter}\NormalTok{(gpa, gpa}\SpecialCharTok{$}\NormalTok{athlete }\SpecialCharTok{==} \ConstantTok{TRUE}\NormalTok{)}
\NormalTok{V\_no\_athlete }\OtherTok{\textless{}{-}} \FunctionTok{filter}\NormalTok{(gpa, gpa}\SpecialCharTok{$}\NormalTok{athlete }\SpecialCharTok{==} \ConstantTok{FALSE}\NormalTok{)}

\NormalTok{h1 }\OtherTok{\textless{}{-}} \FunctionTok{hist}\NormalTok{(V\_athlete}\SpecialCharTok{$}\NormalTok{colgpa ,}\AttributeTok{col=}\StringTok{"lightsalmon"}\NormalTok{,}\AttributeTok{main=}\StringTok{"Histograma Variable colgpa de los athletas"}\NormalTok{)}
\end{Highlighting}
\end{Shaded}

\includegraphics{edumoraglez-PEC2_files/figure-latex/unnamed-chunk-13-1.pdf}

\begin{Shaded}
\begin{Highlighting}[]
\NormalTok{h2 }\OtherTok{\textless{}{-}} \FunctionTok{hist}\NormalTok{(V\_no\_athlete}\SpecialCharTok{$}\NormalTok{colgpa ,}\AttributeTok{col=}\StringTok{"lightsalmon"}\NormalTok{,}\AttributeTok{main=}\StringTok{"Histograma Variable colgpa de NO los athletas"}\NormalTok{)}
\end{Highlighting}
\end{Shaded}

\includegraphics{edumoraglez-PEC2_files/figure-latex/unnamed-chunk-13-2.pdf}

A nivel visual, nos damos cuenta de que las medias de los estudiantes no
deportistas son mayores de los que sí lo son.

\hypertarget{funciuxf3n-para-el-contraste-de-medias}{%
\subsection{Función para el contraste de
medias}\label{funciuxf3n-para-el-contraste-de-medias}}

\begin{Shaded}
\begin{Highlighting}[]
\NormalTok{funcion\_contraste\_medias }\OtherTok{\textless{}{-}} \ControlFlowTok{function}\NormalTok{( x1, x2, }\AttributeTok{CL=}\FloatTok{0.95}\NormalTok{, }\AttributeTok{Var\_Igual=}\ConstantTok{TRUE}\NormalTok{, }\AttributeTok{metodo\_calculo=}\StringTok{"two.sided"}\NormalTok{ )\{}
  
\NormalTok{  media\_1}\OtherTok{\textless{}{-}}\FunctionTok{mean}\NormalTok{(x1)}
\NormalTok{  length\_1}\OtherTok{\textless{}{-}}\FunctionTok{length}\NormalTok{(x1)}
\NormalTok{  sd1}\OtherTok{\textless{}{-}}\FunctionTok{sd}\NormalTok{(x1)}
  
\NormalTok{  media\_2}\OtherTok{\textless{}{-}}\FunctionTok{mean}\NormalTok{(x2)}
\NormalTok{  length\_2}\OtherTok{\textless{}{-}}\FunctionTok{length}\NormalTok{(x2)}
\NormalTok{  sd2}\OtherTok{\textless{}{-}}\FunctionTok{sd}\NormalTok{(x2)}
  
  
  \ControlFlowTok{if}\NormalTok{ (Var\_Igual}\SpecialCharTok{==}\ConstantTok{TRUE}\NormalTok{)\{}
  
\NormalTok{      raiz }\OtherTok{\textless{}{-}}\FunctionTok{sqrt}\NormalTok{( ((length\_1}\DecValTok{{-}1}\NormalTok{)}\SpecialCharTok{*}\NormalTok{sd1}\SpecialCharTok{\^{}}\DecValTok{2} \SpecialCharTok{+}\NormalTok{ (length\_2}\DecValTok{{-}1}\NormalTok{)}\SpecialCharTok{*}\NormalTok{sd2}\SpecialCharTok{\^{}}\DecValTok{2}\NormalTok{ )}\SpecialCharTok{/}\NormalTok{(length\_1}\SpecialCharTok{+}\NormalTok{length\_2}\DecValTok{{-}2}\NormalTok{) )}
\NormalTok{      Sb }\OtherTok{\textless{}{-}}\NormalTok{ raiz}\SpecialCharTok{*}\FunctionTok{sqrt}\NormalTok{(}\DecValTok{1}\SpecialCharTok{/}\NormalTok{length\_1 }\SpecialCharTok{+} \DecValTok{1}\SpecialCharTok{/}\NormalTok{length\_2)}
\NormalTok{      df}\OtherTok{\textless{}{-}}\NormalTok{(length\_1 }\SpecialCharTok{{-}} \DecValTok{1}\NormalTok{) }\SpecialCharTok{+}\NormalTok{ (length\_2 }\SpecialCharTok{{-}} \DecValTok{1}\NormalTok{)}
      
\NormalTok{  \}}\ControlFlowTok{else}\NormalTok{\{}
  
\NormalTok{      Sb }\OtherTok{\textless{}{-}} \FunctionTok{sqrt}\NormalTok{( sd1}\SpecialCharTok{\^{}}\DecValTok{2}\SpecialCharTok{/}\NormalTok{length\_1 }\SpecialCharTok{+}\NormalTok{ sd2}\SpecialCharTok{\^{}}\DecValTok{2}\SpecialCharTok{/}\NormalTok{length\_2 )}
\NormalTok{      denominador }\OtherTok{\textless{}{-}}\NormalTok{ ( (sd1}\SpecialCharTok{\^{}}\DecValTok{2}\SpecialCharTok{/}\NormalTok{length\_1)}\SpecialCharTok{\^{}}\DecValTok{2}\SpecialCharTok{/}\NormalTok{(length\_1}\DecValTok{{-}1}\NormalTok{) }\SpecialCharTok{+}\NormalTok{ (sd2}\SpecialCharTok{\^{}}\DecValTok{2}\SpecialCharTok{/}\NormalTok{length\_2)}\SpecialCharTok{\^{}}\DecValTok{2}\SpecialCharTok{/}\NormalTok{(length\_2}\DecValTok{{-}1}\NormalTok{))}
\NormalTok{      df }\OtherTok{\textless{}{-}}\NormalTok{ ( (sd1}\SpecialCharTok{\^{}}\DecValTok{2}\SpecialCharTok{/}\NormalTok{length\_1 }\SpecialCharTok{+}\NormalTok{ sd2}\SpecialCharTok{\^{}}\DecValTok{2}\SpecialCharTok{/}\NormalTok{length\_2)}\SpecialCharTok{\^{}}\DecValTok{2}\NormalTok{ ) }\SpecialCharTok{/}\NormalTok{ denominador}
      
\NormalTok{  \}}

\NormalTok{  alfa }\OtherTok{\textless{}{-}}\NormalTok{ (}\DecValTok{1}\SpecialCharTok{{-}}\NormalTok{CL)}
\NormalTok{  t}\OtherTok{\textless{}{-}}\NormalTok{ (media\_1}\SpecialCharTok{{-}}\NormalTok{media\_2) }\SpecialCharTok{/}\NormalTok{ Sb}
  
  
  \ControlFlowTok{if}\NormalTok{ (metodo\_calculo}\SpecialCharTok{==}\StringTok{"two.sided"}\NormalTok{)\{}
  
\NormalTok{      tcritical }\OtherTok{\textless{}{-}} \FunctionTok{qt}\NormalTok{( alfa}\SpecialCharTok{/}\DecValTok{2}\NormalTok{, df, }\AttributeTok{lower.tail=}\ConstantTok{FALSE}\NormalTok{ )}
\NormalTok{      pvalue}\OtherTok{\textless{}{-}}\FunctionTok{pt}\NormalTok{( }\FunctionTok{abs}\NormalTok{(t), df, }\AttributeTok{lower.tail=}\ConstantTok{FALSE}\NormalTok{ )}\SpecialCharTok{*}\DecValTok{2}
      
\NormalTok{  \}}
  
  \ControlFlowTok{if}\NormalTok{ (metodo\_calculo}\SpecialCharTok{==}\StringTok{"greater"}\NormalTok{)\{}
  
\NormalTok{      tcritical }\OtherTok{\textless{}{-}} \FunctionTok{qt}\NormalTok{( alfa, df, }\AttributeTok{lower.tail=}\ConstantTok{FALSE}\NormalTok{ )}
\NormalTok{      pvalue}\OtherTok{\textless{}{-}}\FunctionTok{pt}\NormalTok{( t, df, }\AttributeTok{lower.tail=}\ConstantTok{FALSE}\NormalTok{ )  }
      
\NormalTok{  \}}
  
  \ControlFlowTok{if}\NormalTok{ (metodo\_calculo}\SpecialCharTok{==}\StringTok{"less"}\NormalTok{)\{}
  
\NormalTok{      tcritical }\OtherTok{\textless{}{-}} \FunctionTok{qt}\NormalTok{( alfa, df, }\AttributeTok{lower.tail=}\ConstantTok{TRUE}\NormalTok{ )}
\NormalTok{      pvalue}\OtherTok{\textless{}{-}}\FunctionTok{pt}\NormalTok{( t, df, }\AttributeTok{lower.tail=}\ConstantTok{TRUE}\NormalTok{ )}
        
\NormalTok{  \}}
  
  
\NormalTok{  resultados }\OtherTok{\textless{}{-}}\FunctionTok{data.frame}\NormalTok{(t,tcritical,pvalue)}

  \FunctionTok{return}\NormalTok{ (resultados)}

\NormalTok{\}}
\end{Highlighting}
\end{Shaded}

\hypertarget{pregunta-de-investigaciuxf3n}{%
\subsection{Pregunta de
investigación}\label{pregunta-de-investigaciuxf3n}}

¿Ser atleta influye en la nota?

\hypertarget{hipuxf3tesis-nula-y-la-alternativa}{%
\subsection{Hipótesis nula y la
alternativa}\label{hipuxf3tesis-nula-y-la-alternativa}}

H0 : µNotaAthleta = µNotaNoAthleta

H1 : µNotaAthleta \textless{} µNotaNoAthleta

\hypertarget{justificaciuxf3n-del-test-a-aplicar}{%
\subsection{Justificación del test a
aplicar}\label{justificaciuxf3n-del-test-a-aplicar}}

La prueba para aplicar es la diferencia de medias, ya que tenemos dos
grupos bien diferenciados, los deportistas y los no. Y al querer
comprobar si la diferencia de notas influye a factor del deporte, esta
prueba puede mostrar un resultado adecuado que pueda contestar la
pregunta.

\begin{Shaded}
\begin{Highlighting}[]
\CommentTok{\#Comprobamos si las varianzas son iguales}
\FunctionTok{var.test}\NormalTok{( gpa}\SpecialCharTok{$}\NormalTok{colgpa[gpa}\SpecialCharTok{$}\NormalTok{athlete }\SpecialCharTok{==} \ConstantTok{TRUE}\NormalTok{], gpa}\SpecialCharTok{$}\NormalTok{colgpa[gpa}\SpecialCharTok{$}\NormalTok{athlete }\SpecialCharTok{==} \ConstantTok{FALSE}\NormalTok{] )}
\end{Highlighting}
\end{Shaded}

\begin{verbatim}
## 
##  F test to compare two variances
## 
## data:  gpa$colgpa[gpa$athlete == TRUE] and gpa$colgpa[gpa$athlete == FALSE]
## F = 0.82199, num df = 193, denom df = 3942, p-value = 0.07287
## alternative hypothesis: true ratio of variances is not equal to 1
## 95 percent confidence interval:
##  0.6762059 1.0186147
## sample estimates:
## ratio of variances 
##          0.8219902
\end{verbatim}

El resultado del test es un valor p \textgreater{} 0.05, por lo que no
se descarta la diferencia de varianzas de las poblaciones atletas y no
atletas. Por lo que el tipo de test es con dos muestras distintas y
varianzas iguales.

\hypertarget{cuxe1lculo}{%
\subsection{Cálculo}\label{cuxe1lculo}}

\begin{Shaded}
\begin{Highlighting}[]
\CommentTok{\#Uso funcion implementada}
\FunctionTok{funcion\_contraste\_medias}\NormalTok{( gpa}\SpecialCharTok{$}\NormalTok{colgpa[gpa}\SpecialCharTok{$}\NormalTok{athlete }\SpecialCharTok{==} \ConstantTok{TRUE}\NormalTok{], gpa}\SpecialCharTok{$}\NormalTok{colgpa[gpa}\SpecialCharTok{$}\NormalTok{athlete }\SpecialCharTok{==} \ConstantTok{FALSE}\NormalTok{], }\AttributeTok{Var\_Igual=}\ConstantTok{TRUE}\NormalTok{, }\AttributeTok{metodo\_calculo =} \StringTok{"two.sided"}\NormalTok{)}
\end{Highlighting}
\end{Shaded}

\begin{verbatim}
##           t tcritical       pvalue
## 1 -5.910309  1.960538 3.689891e-09
\end{verbatim}

\begin{Shaded}
\begin{Highlighting}[]
\CommentTok{\#Validación de la función}
\FunctionTok{t.test}\NormalTok{( gpa}\SpecialCharTok{$}\NormalTok{colgpa[gpa}\SpecialCharTok{$}\NormalTok{athlete }\SpecialCharTok{==} \ConstantTok{TRUE}\NormalTok{], gpa}\SpecialCharTok{$}\NormalTok{colgpa[gpa}\SpecialCharTok{$}\NormalTok{athlete }\SpecialCharTok{==} \ConstantTok{FALSE}\NormalTok{], }\AttributeTok{var.equal=}\ConstantTok{TRUE}\NormalTok{, }\AttributeTok{alternative =} \StringTok{"two.sided"}\NormalTok{)}
\end{Highlighting}
\end{Shaded}

\begin{verbatim}
## 
##  Two Sample t-test
## 
## data:  gpa$colgpa[gpa$athlete == TRUE] and gpa$colgpa[gpa$athlete == FALSE]
## t = -5.9103, df = 4135, p-value = 3.69e-09
## alternative hypothesis: true difference in means is not equal to 0
## 95 percent confidence interval:
##  -0.3792088 -0.1902956
## sample estimates:
## mean of x mean of y 
##  2.382732  2.667484
\end{verbatim}

\hypertarget{interpretaciuxf3n-del-test}{%
\subsection{Interpretación del test}\label{interpretaciuxf3n-del-test}}

El valor crítico para un nivel de confianza del 95\% es 1.960538 y el
valor observado es -5.910309. Con estos resultados en una zona de
rechazo de la hipótesis nula y podemos concluir que los estudiantes no
atletas tienen mejor nota que los que lo son.

Se concluye lo mismo con el valor P= 3.689891e-09, que es muy inferior
al valor de alfa que es 0.05, comparando este resultado con los
intervalos de confianza de los atletas y los que no lo son, llegamos a
la misma conclusión en los dos casos.

\hypertarget{las-mujeres-tienen-mejor-nota-que-los-hombres}{%
\section{¿Las mujeres tienen mejor nota que los
hombres?}\label{las-mujeres-tienen-mejor-nota-que-los-hombres}}

\hypertarget{anuxe1lisis-visual-1}{%
\subsection{Análisis visual}\label{anuxe1lisis-visual-1}}

\begin{Shaded}
\begin{Highlighting}[]
\NormalTok{V\_female }\OtherTok{\textless{}{-}} \FunctionTok{filter}\NormalTok{(gpa, gpa}\SpecialCharTok{$}\NormalTok{female }\SpecialCharTok{==} \ConstantTok{TRUE}\NormalTok{)}
\NormalTok{V\_no\_female }\OtherTok{\textless{}{-}} \FunctionTok{filter}\NormalTok{(gpa, gpa}\SpecialCharTok{$}\NormalTok{female }\SpecialCharTok{==} \ConstantTok{FALSE}\NormalTok{)}

\NormalTok{h1 }\OtherTok{\textless{}{-}} \FunctionTok{hist}\NormalTok{(V\_female}\SpecialCharTok{$}\NormalTok{colgpa ,}\AttributeTok{col=}\StringTok{"lightsalmon"}\NormalTok{,}\AttributeTok{main=}\StringTok{"Histograma Variable colgpa de las Mujeres"}\NormalTok{)}
\end{Highlighting}
\end{Shaded}

\includegraphics{edumoraglez-PEC2_files/figure-latex/unnamed-chunk-17-1.pdf}

\begin{Shaded}
\begin{Highlighting}[]
\NormalTok{h2 }\OtherTok{\textless{}{-}} \FunctionTok{hist}\NormalTok{(V\_no\_female}\SpecialCharTok{$}\NormalTok{colgpa ,}\AttributeTok{col=}\StringTok{"lightsalmon"}\NormalTok{,}\AttributeTok{main=}\StringTok{"Histograma Variable colgpa de los Hombres"}\NormalTok{)}
\end{Highlighting}
\end{Shaded}

\includegraphics{edumoraglez-PEC2_files/figure-latex/unnamed-chunk-17-2.pdf}

A nivel visual, nos damos cuenta de que las medias de los hombres son un
poco mejor que las de las mujeres, aunque el cambio (de manera visual)
no es muy significativo.

\hypertarget{funciuxf3n-para-el-contraste-de-medias-1}{%
\subsection{Función para el contraste de
medias}\label{funciuxf3n-para-el-contraste-de-medias-1}}

Usamos la misma función que para la pregunta anterior

\hypertarget{pregunta-de-investigaciuxf3n-1}{%
\subsection{Pregunta de
investigación}\label{pregunta-de-investigaciuxf3n-1}}

¿Las mujeres tienen mejor nota que los hombres?

\hypertarget{hipuxf3tesis-nula-y-la-alternativa-1}{%
\subsection{Hipótesis nula y la
alternativa}\label{hipuxf3tesis-nula-y-la-alternativa-1}}

H0 : µNotaMujer = µNotaHombre

H1 : µNotaMujer \textless= µNotaHombre

\hypertarget{justificaciuxf3n-del-test-a-aplicar-al-95}\label{justificaciuxf3n-del-test-a-aplicar-al-95}}

\begin{Shaded}
\begin{Highlighting}[]
\CommentTok{\#Comprobamos si las varianzas son iguales}
\FunctionTok{var.test}\NormalTok{( gpa}\SpecialCharTok{$}\NormalTok{colgpa[gpa}\SpecialCharTok{$}\NormalTok{female }\SpecialCharTok{==} \ConstantTok{TRUE}\NormalTok{], gpa}\SpecialCharTok{$}\NormalTok{colgpa[gpa}\SpecialCharTok{$}\NormalTok{female }\SpecialCharTok{==} \ConstantTok{FALSE}\NormalTok{] )}
\end{Highlighting}
\end{Shaded}

\begin{verbatim}
## 
##  F test to compare two variances
## 
## data:  gpa$colgpa[gpa$female == TRUE] and gpa$colgpa[gpa$female == FALSE]
## F = 0.82757, num df = 1859, denom df = 2276, p-value = 2.024e-05
## alternative hypothesis: true ratio of variances is not equal to 1
## 95 percent confidence interval:
##  0.7590051 0.9026724
## sample estimates:
## ratio of variances 
##          0.8275687
\end{verbatim}

El resultado del test es un valor p \textless{} 0.001, por lo que se
descarta la igualdad de varianzas de las poblaciones Mujeres y Hobres.
Por lo que el tipo de test es con dos muestras distintas y varianzas
diferentes.

\hypertarget{cuxe1lculo-a-95}\label{cuxe1lculo-a-95}}

\begin{Shaded}
\begin{Highlighting}[]
\CommentTok{\#Uso funcion implementada}
\FunctionTok{funcion\_contraste\_medias}\NormalTok{( gpa}\SpecialCharTok{$}\NormalTok{colgpa[gpa}\SpecialCharTok{$}\NormalTok{female }\SpecialCharTok{==} \ConstantTok{TRUE}\NormalTok{], gpa}\SpecialCharTok{$}\NormalTok{colgpa[gpa}\SpecialCharTok{$}\NormalTok{female }\SpecialCharTok{==} \ConstantTok{FALSE}\NormalTok{], }\AttributeTok{Var\_Igual=}\ConstantTok{FALSE}\NormalTok{, }\AttributeTok{metodo\_calculo =} \StringTok{"two.sided"}\NormalTok{)}
\end{Highlighting}
\end{Shaded}

\begin{verbatim}
##          t tcritical       pvalue
## 1 7.078735  1.960545 1.704394e-12
\end{verbatim}

\begin{Shaded}
\begin{Highlighting}[]
\CommentTok{\#Validación de la función}
\FunctionTok{t.test}\NormalTok{( gpa}\SpecialCharTok{$}\NormalTok{colgpa[gpa}\SpecialCharTok{$}\NormalTok{female }\SpecialCharTok{==} \ConstantTok{TRUE}\NormalTok{], gpa}\SpecialCharTok{$}\NormalTok{colgpa[gpa}\SpecialCharTok{$}\NormalTok{female }\SpecialCharTok{==} \ConstantTok{FALSE}\NormalTok{], }\AttributeTok{var.equal=}\ConstantTok{FALSE}\NormalTok{, }\AttributeTok{alternative =} \StringTok{"two.sided"}\NormalTok{)}
\end{Highlighting}
\end{Shaded}

\begin{verbatim}
## 
##  Welch Two Sample t-test
## 
## data:  gpa$colgpa[gpa$female == TRUE] and gpa$colgpa[gpa$female == FALSE]
## t = 7.0787, df = 4087.4, p-value = 1.704e-12
## alternative hypothesis: true difference in means is not equal to 0
## 95 percent confidence interval:
##  0.1036283 0.1830188
## sample estimates:
## mean of x mean of y 
##  2.733016  2.589693
\end{verbatim}

\hypertarget{interpretaciuxf3n-del-test-a-95}\label{interpretaciuxf3n-del-test-a-95}}

El valor crítico para un nivel de confianza del 95\% es 1.960545 y el
valor observado es 7.078735 . Con estos resultados en una zona de
rechazo de la hipótesis nula y podemos concluir que los estudiantes
hombres tienen mejores notas que las mujeres.

Se concluye lo mismo con el valor P= 1.704394e-12, que es muy inferior
al valor de alfa que es 0.05, comparando este resultado con los
intervalos de confianza de las mujeres y los hombres, llegamos a la
misma conclusión en los dos casos.

\hypertarget{justificaciuxf3n-del-test-a-aplicar-al-90}\label{justificaciuxf3n-del-test-a-aplicar-al-90}}

\begin{Shaded}
\begin{Highlighting}[]
\CommentTok{\#Comprobamos si las varianzas son iguales}
\FunctionTok{var.test}\NormalTok{( gpa}\SpecialCharTok{$}\NormalTok{colgpa[gpa}\SpecialCharTok{$}\NormalTok{female }\SpecialCharTok{==} \ConstantTok{TRUE}\NormalTok{], gpa}\SpecialCharTok{$}\NormalTok{colgpa[gpa}\SpecialCharTok{$}\NormalTok{female }\SpecialCharTok{==} \ConstantTok{FALSE}\NormalTok{], }\AttributeTok{conf.level =} \FloatTok{0.9}\NormalTok{ )}
\end{Highlighting}
\end{Shaded}

\begin{verbatim}
## 
##  F test to compare two variances
## 
## data:  gpa$colgpa[gpa$female == TRUE] and gpa$colgpa[gpa$female == FALSE]
## F = 0.82757, num df = 1859, denom df = 2276, p-value = 2.024e-05
## alternative hypothesis: true ratio of variances is not equal to 1
## 90 percent confidence interval:
##  0.7696299 0.8901446
## sample estimates:
## ratio of variances 
##          0.8275687
\end{verbatim}

El resultado del test es un valor p \textless{} 0.001, por lo que se
descarta la igualdad de varianzas de las poblaciones Mujeres y Hobres.
Por lo que el tipo de test es con dos muestras distintas y varianzas
diferentes.

\hypertarget{cuxe1lculo-a-90}\label{cuxe1lculo-a-90}}

\begin{Shaded}
\begin{Highlighting}[]
\CommentTok{\#Uso funcion implementada}
\FunctionTok{funcion\_contraste\_medias}\NormalTok{( gpa}\SpecialCharTok{$}\NormalTok{colgpa[gpa}\SpecialCharTok{$}\NormalTok{female }\SpecialCharTok{==} \ConstantTok{TRUE}\NormalTok{], gpa}\SpecialCharTok{$}\NormalTok{colgpa[gpa}\SpecialCharTok{$}\NormalTok{female }\SpecialCharTok{==} \ConstantTok{FALSE}\NormalTok{], }\AttributeTok{CL=}\FloatTok{0.90}\NormalTok{, }\AttributeTok{Var\_Igual=}\ConstantTok{FALSE}\NormalTok{, }\AttributeTok{metodo\_calculo =} \StringTok{"two.sided"}\NormalTok{)}
\end{Highlighting}
\end{Shaded}

\begin{verbatim}
##          t tcritical       pvalue
## 1 7.078735  1.645227 1.704394e-12
\end{verbatim}

\begin{Shaded}
\begin{Highlighting}[]
\CommentTok{\#Validación de la función}
\FunctionTok{t.test}\NormalTok{( gpa}\SpecialCharTok{$}\NormalTok{colgpa[gpa}\SpecialCharTok{$}\NormalTok{female }\SpecialCharTok{==} \ConstantTok{TRUE}\NormalTok{], gpa}\SpecialCharTok{$}\NormalTok{colgpa[gpa}\SpecialCharTok{$}\NormalTok{female }\SpecialCharTok{==} \ConstantTok{FALSE}\NormalTok{], }\AttributeTok{var.equal=}\ConstantTok{FALSE}\NormalTok{, }\AttributeTok{conf.level =} \FloatTok{0.9}\NormalTok{, }\AttributeTok{alternative =} \StringTok{"two.sided"}\NormalTok{)}
\end{Highlighting}
\end{Shaded}

\begin{verbatim}
## 
##  Welch Two Sample t-test
## 
## data:  gpa$colgpa[gpa$female == TRUE] and gpa$colgpa[gpa$female == FALSE]
## t = 7.0787, df = 4087.4, p-value = 1.704e-12
## alternative hypothesis: true difference in means is not equal to 0
## 90 percent confidence interval:
##  0.1100126 0.1766345
## sample estimates:
## mean of x mean of y 
##  2.733016  2.589693
\end{verbatim}

\hypertarget{interpretaciuxf3n-del-test-a-90}\label{interpretaciuxf3n-del-test-a-90}}

El valor crítico para un nivel de confianza del 90\% es 1.645227 y el
valor observado es 7.078735 . Con estos resultados en una zona de
rechazo de la hipótesis nula y podemos concluir que los estudiantes
hombres tienen mejores notas que las mujeres.

Se concluye lo mismo con el valor P= 1.704394e-12, que es muy inferior
al valor de alfa que es 0.1, comparando este resultado con los
intervalos de confianza de las mujeres y los hombres, llegamos a la
misma conclusión en los dos casos.

\hypertarget{hay-diferencias-en-la-nota-seguxfan-la-raza}{%
\section{¿Hay diferencias en la nota según la
raza?}\label{hay-diferencias-en-la-nota-seguxfan-la-raza}}

\hypertarget{anuxe1lisis-visual-2}{%
\subsection{Análisis visual}\label{anuxe1lisis-visual-2}}

\begin{Shaded}
\begin{Highlighting}[]
\NormalTok{V\_black }\OtherTok{\textless{}{-}} \FunctionTok{filter}\NormalTok{(gpa, gpa}\SpecialCharTok{$}\NormalTok{black }\SpecialCharTok{==} \ConstantTok{TRUE}\NormalTok{)}
\NormalTok{V\_white }\OtherTok{\textless{}{-}} \FunctionTok{filter}\NormalTok{(gpa, gpa}\SpecialCharTok{$}\NormalTok{white }\SpecialCharTok{==} \ConstantTok{TRUE}\NormalTok{)}

\NormalTok{h1 }\OtherTok{\textless{}{-}} \FunctionTok{hist}\NormalTok{(V\_black}\SpecialCharTok{$}\NormalTok{colgpa ,}\AttributeTok{col=}\StringTok{"lightsalmon"}\NormalTok{,}\AttributeTok{main=}\StringTok{"Histograma Variable colgpa de las personas Negras"}\NormalTok{)}
\end{Highlighting}
\end{Shaded}

\includegraphics{edumoraglez-PEC2_files/figure-latex/unnamed-chunk-22-1.pdf}

\begin{Shaded}
\begin{Highlighting}[]
\NormalTok{h2 }\OtherTok{\textless{}{-}} \FunctionTok{hist}\NormalTok{(V\_white}\SpecialCharTok{$}\NormalTok{colgpa ,}\AttributeTok{col=}\StringTok{"lightsalmon"}\NormalTok{,}\AttributeTok{main=}\StringTok{"Histograma Variable colgpa de las personas Blancas"}\NormalTok{)}
\end{Highlighting}
\end{Shaded}

\includegraphics{edumoraglez-PEC2_files/figure-latex/unnamed-chunk-22-2.pdf}

A nivel visual, nos damos cuenta de que las medias de los estudiantes de
raza blanca son mayores de los de raza negra.

\hypertarget{funciuxf3n-para-el-contraste-de-medias-2}{%
\subsection{Función para el contraste de
medias}\label{funciuxf3n-para-el-contraste-de-medias-2}}

Usamos la misma función que para la pregunta anterior

\hypertarget{pregunta-de-investigaciuxf3n-2}{%
\subsection{Pregunta de
investigación}\label{pregunta-de-investigaciuxf3n-2}}

¿Hay diferencias en la nota según la raza?

\hypertarget{hipuxf3tesis-nula-y-la-alternativa-2}{%
\subsection{Hipótesis nula y la
alternativa}\label{hipuxf3tesis-nula-y-la-alternativa-2}}

H0 : µNotaRazaNegra = µNotaRazaBlanca

H1 : µNotaRazaNegra != µNotaRazaBlanca

\hypertarget{justificaciuxf3n-del-test-a-aplicar-1}{%
\subsection{Justificación del test a
aplicar}\label{justificaciuxf3n-del-test-a-aplicar-1}}

\begin{Shaded}
\begin{Highlighting}[]
\CommentTok{\#Comprobamos si las varianzas son iguales}
\FunctionTok{var.test}\NormalTok{( gpa}\SpecialCharTok{$}\NormalTok{colgpa[gpa}\SpecialCharTok{$}\NormalTok{white }\SpecialCharTok{==} \ConstantTok{TRUE}\NormalTok{], gpa}\SpecialCharTok{$}\NormalTok{colgpa[gpa}\SpecialCharTok{$}\NormalTok{black }\SpecialCharTok{==} \ConstantTok{TRUE}\NormalTok{] )}
\end{Highlighting}
\end{Shaded}

\begin{verbatim}
## 
##  F test to compare two variances
## 
## data:  gpa$colgpa[gpa$white == TRUE] and gpa$colgpa[gpa$black == TRUE]
## F = 1.127, num df = 3828, denom df = 228, p-value = 0.2343
## alternative hypothesis: true ratio of variances is not equal to 1
## 95 percent confidence interval:
##  0.9250878 1.3509805
## sample estimates:
## ratio of variances 
##           1.126968
\end{verbatim}

El resultado del test es un valor p \textgreater{} 0.001, por lo que NO
se descarta la igualdad de varianzas de las poblaciones de Negros y
Blancos. Por lo que el tipo de test es con dos muestras distintas y
varianzas iguales.

\hypertarget{cuxe1lculo-1}{%
\subsection{Cálculo}\label{cuxe1lculo-1}}

\begin{Shaded}
\begin{Highlighting}[]
\CommentTok{\#Uso funcion implementada}
\FunctionTok{funcion\_contraste\_medias}\NormalTok{( gpa}\SpecialCharTok{$}\NormalTok{colgpa[gpa}\SpecialCharTok{$}\NormalTok{white }\SpecialCharTok{==} \ConstantTok{TRUE}\NormalTok{], gpa}\SpecialCharTok{$}\NormalTok{colgpa[gpa}\SpecialCharTok{$}\NormalTok{black }\SpecialCharTok{==} \ConstantTok{TRUE}\NormalTok{], }\AttributeTok{Var\_Igual=}\ConstantTok{TRUE}\NormalTok{, }\AttributeTok{metodo\_calculo =} \StringTok{"two.sided"}\NormalTok{)}
\end{Highlighting}
\end{Shaded}

\begin{verbatim}
##          t tcritical      pvalue
## 1 9.559319  1.960549 1.99014e-21
\end{verbatim}

\begin{Shaded}
\begin{Highlighting}[]
\CommentTok{\#Validación de la función}
\FunctionTok{t.test}\NormalTok{(gpa}\SpecialCharTok{$}\NormalTok{colgpa[gpa}\SpecialCharTok{$}\NormalTok{white }\SpecialCharTok{==} \ConstantTok{TRUE}\NormalTok{], gpa}\SpecialCharTok{$}\NormalTok{colgpa[gpa}\SpecialCharTok{$}\NormalTok{black }\SpecialCharTok{==} \ConstantTok{TRUE}\NormalTok{], }\AttributeTok{var.equal=}\ConstantTok{TRUE}\NormalTok{, }\AttributeTok{alternative =} \StringTok{"two.sided"}\NormalTok{)}
\end{Highlighting}
\end{Shaded}

\begin{verbatim}
## 
##  Two Sample t-test
## 
## data:  gpa$colgpa[gpa$white == TRUE] and gpa$colgpa[gpa$black == TRUE]
## t = 9.5593, df = 4056, p-value < 2.2e-16
## alternative hypothesis: true difference in means is not equal to 0
## 95 percent confidence interval:
##  0.3360979 0.5095302
## sample estimates:
## mean of x mean of y 
##  2.678360  2.255546
\end{verbatim}

\hypertarget{interpretaciuxf3n-del-test-1}{%
\subsection{Interpretación del
test}\label{interpretaciuxf3n-del-test-1}}

El valor crítico para un nivel de confianza del 95\% es 1.960549 y el
valor observado es 9.559319. Con estos resultados en una zona de rechazo
de la hipótesis nula y podemos concluir que los estudiantes blancos
tienen mejores notas que los estudiantes negros.

Se concluye lo mismo con el valor P=1.99014e-21, que es muy inferior al
valor de alfa que es 0.05, comparando este resultado con los intervalos
de confianza de las personas negras y blancas, llegamos a la misma
conclusión en los dos casos.

\hypertarget{proporciuxf3n-de-atletas}{%
\section{Proporción de atletas}\label{proporciuxf3n-de-atletas}}

\hypertarget{anuxe1lisis-visual-3}{%
\subsection{Análisis visual}\label{anuxe1lisis-visual-3}}

\begin{Shaded}
\begin{Highlighting}[]
\NormalTok{tabla }\OtherTok{\textless{}{-}} \FunctionTok{table}\NormalTok{(gpa}\SpecialCharTok{$}\NormalTok{athlete)}

\NormalTok{etiquetas }\OtherTok{\textless{}{-}}\FunctionTok{paste0}\NormalTok{(}\FunctionTok{round}\NormalTok{(}\DecValTok{100} \SpecialCharTok{*}\NormalTok{ tabla}\SpecialCharTok{/}\FunctionTok{sum}\NormalTok{(tabla), }\DecValTok{2}\NormalTok{), }\StringTok{"\%"}\NormalTok{)}

\FunctionTok{pie}\NormalTok{(tabla, }\AttributeTok{labels =}\NormalTok{ etiquetas, }\AttributeTok{main=}\StringTok{"Proporción de personas Athletas entre los estudiantes"}\NormalTok{)}

\FunctionTok{legend}\NormalTok{(}\StringTok{"topleft"}\NormalTok{, }\AttributeTok{legend =} \FunctionTok{c}\NormalTok{(}\StringTok{"TRUE"}\NormalTok{, }\StringTok{"FALSE"}\NormalTok{),}
       \AttributeTok{fill =}  \FunctionTok{c}\NormalTok{(}\StringTok{"White"}\NormalTok{, }\StringTok{"lightblue"}\NormalTok{))}
\end{Highlighting}
\end{Shaded}

\includegraphics{edumoraglez-PEC2_files/figure-latex/unnamed-chunk-25-1.pdf}

A nivel visual, vemos que la proporción de deportista entre lo
estudiantes es menor de los que lo son.

\hypertarget{pregunta-de-investigaciuxf3n-3}{%
\subsection{Pregunta de
investigación}\label{pregunta-de-investigaciuxf3n-3}}

¿La proporción de atletas en la población es inferior al 5\%?

\hypertarget{hipuxf3tesis-nula-y-la-alternativa-3}{%
\subsection{Hipótesis nula y la
alternativa}\label{hipuxf3tesis-nula-y-la-alternativa-3}}

H0 : p \textless= 0.5

H1 : p \textgreater{} 0.5

donde p es la proporción de athletas en la sociedad

\hypertarget{justificaciuxf3n-del-test-a-aplicar-2}{%
\subsection{Justificación del test a
aplicar}\label{justificaciuxf3n-del-test-a-aplicar-2}}

Se debe realizar un test de una muestra sobre la proporción. Se calcula
la proporción de estudiantes que son atletas es menor que la proporción
de los que no lo son. Esta proporción se compara con el valor 0.5.

\hypertarget{realizad-los-cuxe1lculos-del-test}{%
\subsection{Realizad los cálculos del
test}\label{realizad-los-cuxe1lculos-del-test}}

\begin{Shaded}
\begin{Highlighting}[]
\NormalTok{n }\OtherTok{\textless{}{-}} \FunctionTok{nrow}\NormalTok{(gpa)}
\NormalTok{p }\OtherTok{=}\NormalTok{ (}\FunctionTok{sum}\NormalTok{(gpa}\SpecialCharTok{$}\NormalTok{athlete }\SpecialCharTok{==} \ConstantTok{TRUE}\NormalTok{))}\SpecialCharTok{/}\NormalTok{ n;}

\NormalTok{p0 }\OtherTok{\textless{}{-}} \FloatTok{0.5}

\NormalTok{Valor\_Observado }\OtherTok{\textless{}{-}}\NormalTok{ (p }\SpecialCharTok{{-}}\NormalTok{ p0)}\SpecialCharTok{/} \FunctionTok{sqrt}\NormalTok{( p0}\SpecialCharTok{*}\NormalTok{(}\DecValTok{1}\SpecialCharTok{{-}}\NormalTok{p0)}\SpecialCharTok{/}\NormalTok{n)}
\NormalTok{Valor\_Critico }\OtherTok{\textless{}{-}} \FunctionTok{qnorm}\NormalTok{( }\FloatTok{0.05}\NormalTok{, }\AttributeTok{lower.tail=}\ConstantTok{FALSE}\NormalTok{ )}
\NormalTok{pvalue }\OtherTok{\textless{}{-}} \FunctionTok{pnorm}\NormalTok{( Valor\_Observado, }\AttributeTok{lower.tail=}\ConstantTok{FALSE}\NormalTok{ )}

\FunctionTok{data.frame}\NormalTok{(p,Valor\_Observado,Valor\_Critico, pvalue)}
\end{Highlighting}
\end{Shaded}

\begin{verbatim}
##            p Valor_Observado Valor_Critico pvalue
## 1 0.04689388       -58.28713      1.644854      1
\end{verbatim}

\begin{Shaded}
\begin{Highlighting}[]
\CommentTok{\#Comprobación del Test}
\FunctionTok{prop.test}\NormalTok{( }\FunctionTok{sum}\NormalTok{( gpa}\SpecialCharTok{$}\NormalTok{athlete), n, }\AttributeTok{p=}\FloatTok{0.5}\NormalTok{, }\AttributeTok{correct=}\ConstantTok{FALSE}\NormalTok{, }\AttributeTok{alternative=}\StringTok{"greater"}\NormalTok{)}
\end{Highlighting}
\end{Shaded}

\begin{verbatim}
## 
##  1-sample proportions test without continuity correction
## 
## data:  sum(gpa$athlete) out of n, null probability 0.5
## X-squared = 3397.4, df = 1, p-value = 1
## alternative hypothesis: true p is greater than 0.5
## 95 percent confidence interval:
##  0.04177721 1.00000000
## sample estimates:
##          p 
## 0.04689388
\end{verbatim}

\hypertarget{interpretaciuxf3n-del-test-2}{%
\subsection{Interpretación del
test}\label{interpretaciuxf3n-del-test-2}}

El valor crítico para α=0.05 es 1.644854 y el valor observado es
-58.28713. Por tanto, el valor observado se encuentra dentro de la zona
de aceptación y podemos aprobar la hipótesis nula. Se llega a la misma
conclusión con el valor p, que es igual y superior a α=0.05, por lo que
se concluye de que la proporción de estudiantes atletas es menor al 5\%.

\hypertarget{hay-muxe1s-atletas-entre-los-hombres-que-entre-las-mujeres}{%
\section{¿Hay más atletas entre los hombres que entre las
mujeres?}\label{hay-muxe1s-atletas-entre-los-hombres-que-entre-las-mujeres}}

\hypertarget{anuxe1lisis-visual-4}{%
\subsection{Análisis visual}\label{anuxe1lisis-visual-4}}

\begin{Shaded}
\begin{Highlighting}[]
\NormalTok{tabla }\OtherTok{\textless{}{-}} \FunctionTok{table}\NormalTok{(gpa}\SpecialCharTok{$}\NormalTok{female[gpa}\SpecialCharTok{$}\NormalTok{athlete}\SpecialCharTok{==}\ConstantTok{TRUE}\NormalTok{])}

\NormalTok{etiquetas }\OtherTok{\textless{}{-}}\FunctionTok{paste0}\NormalTok{(}\FunctionTok{round}\NormalTok{(}\DecValTok{100} \SpecialCharTok{*}\NormalTok{ tabla}\SpecialCharTok{/}\FunctionTok{sum}\NormalTok{(tabla), }\DecValTok{2}\NormalTok{), }\StringTok{"\%"}\NormalTok{)}

\FunctionTok{pie}\NormalTok{(tabla, }\AttributeTok{labels =}\NormalTok{ etiquetas, }\AttributeTok{main=}\StringTok{"Proporción de personas Atletas según el sexo"}\NormalTok{)}
\FunctionTok{legend}\NormalTok{(}\StringTok{"topleft"}\NormalTok{, }\AttributeTok{legend =} \FunctionTok{c}\NormalTok{(}\StringTok{"HOMBRES"}\NormalTok{, }\StringTok{"MUJERES"}\NormalTok{),}
       \AttributeTok{fill =}  \FunctionTok{c}\NormalTok{(}\StringTok{"White"}\NormalTok{, }\StringTok{"lightblue"}\NormalTok{))}
\end{Highlighting}
\end{Shaded}

\includegraphics{edumoraglez-PEC2_files/figure-latex/unnamed-chunk-27-1.pdf}

A nivel visual, vemos que la proporción de deportistas entre las mujeres
es menor que la de los hombres.

\hypertarget{pregunta-de-investigaciuxf3n-4}{%
\subsection{Pregunta de
investigación}\label{pregunta-de-investigaciuxf3n-4}}

¿Hay más atletas entre los hombres que entre las mujeres?

\hypertarget{hipuxf3tesis-nula-y-la-alternativa-4}{%
\subsection{Hipótesis nula y la
alternativa}\label{hipuxf3tesis-nula-y-la-alternativa-4}}

H0 : Phombres \textless{} Pmujeres

H1 : Phombres \textgreater{} Pmujeres

donde P es la proporción de athletas

\hypertarget{justificaciuxf3n-del-test-a-aplicar-3}{%
\subsection{Justificación del test a
aplicar}\label{justificaciuxf3n-del-test-a-aplicar-3}}

Test para la diferencia de dos proporciones, para ello se calculará la
proporción entre los hombres y mujeres, y si son deportistas o no lo
son. Al tener dos proporciones, se compara si la primera es
significativamente diferente de la segunda, siendo un test bilateral.

\hypertarget{realizad-los-cuxe1lculos-del-test-1}{%
\subsection{Realizad los cálculos del
test}\label{realizad-los-cuxe1lculos-del-test-1}}

\begin{Shaded}
\begin{Highlighting}[]
\NormalTok{Pob.female }\OtherTok{\textless{}{-}}\NormalTok{ gpa[gpa}\SpecialCharTok{$}\NormalTok{female}\SpecialCharTok{==}\ConstantTok{TRUE}\NormalTok{,]}
\NormalTok{Pob.nfemale }\OtherTok{\textless{}{-}}\NormalTok{ gpa[gpa}\SpecialCharTok{$}\NormalTok{female}\SpecialCharTok{==}\ConstantTok{FALSE}\NormalTok{,]}
\NormalTok{n1 }\OtherTok{\textless{}{-}}\FunctionTok{nrow}\NormalTok{(Pob.female)}
\NormalTok{n2}\OtherTok{\textless{}{-}}\FunctionTok{nrow}\NormalTok{(Pob.nfemale)}
\NormalTok{p1 }\OtherTok{\textless{}{-}} \FunctionTok{sum}\NormalTok{(Pob.female}\SpecialCharTok{$}\NormalTok{athlete[Pob.female}\SpecialCharTok{$}\NormalTok{athlete }\SpecialCharTok{==} \ConstantTok{TRUE}\NormalTok{]) }\SpecialCharTok{/}\NormalTok{ n1}
\NormalTok{p2 }\OtherTok{\textless{}{-}} \FunctionTok{sum}\NormalTok{(Pob.nfemale}\SpecialCharTok{$}\NormalTok{athlete[Pob.nfemale}\SpecialCharTok{$}\NormalTok{athlete }\SpecialCharTok{==} \ConstantTok{TRUE}\NormalTok{]) }\SpecialCharTok{/}\NormalTok{ n2}


\NormalTok{p}\OtherTok{\textless{}{-}}\NormalTok{ (n1}\SpecialCharTok{*}\NormalTok{p1 }\SpecialCharTok{+}\NormalTok{ n2}\SpecialCharTok{*}\NormalTok{p2) }\SpecialCharTok{/}\NormalTok{(n1}\SpecialCharTok{+}\NormalTok{n2)}
\NormalTok{z}\OtherTok{\textless{}{-}}\NormalTok{ (p1}\SpecialCharTok{{-}}\NormalTok{p2) }\SpecialCharTok{/} \FunctionTok{sqrt}\NormalTok{( p}\SpecialCharTok{*}\NormalTok{(}\DecValTok{1}\SpecialCharTok{{-}}\NormalTok{p)}\SpecialCharTok{*}\NormalTok{(}\DecValTok{1}\SpecialCharTok{/}\NormalTok{n1 }\SpecialCharTok{+} \DecValTok{1}\SpecialCharTok{/}\NormalTok{n2))}
\NormalTok{zcrit }\OtherTok{\textless{}{-}} \FunctionTok{qnorm}\NormalTok{(}\FloatTok{0.025}\NormalTok{)}
\NormalTok{pvalue }\OtherTok{\textless{}{-}} \FunctionTok{pnorm}\NormalTok{(}\FunctionTok{abs}\NormalTok{(z),}\AttributeTok{lower.tail=}\ConstantTok{FALSE}\NormalTok{)}\SpecialCharTok{*}\DecValTok{2}


\FunctionTok{data.frame}\NormalTok{(z, zcrit, pvalue)}
\end{Highlighting}
\end{Shaded}

\begin{verbatim}
##           z     zcrit       pvalue
## 1 -6.241964 -1.959964 4.321099e-10
\end{verbatim}

\begin{Shaded}
\begin{Highlighting}[]
\CommentTok{\#Validación femaleando prop.test}
\NormalTok{success }\OtherTok{\textless{}{-}} \FunctionTok{c}\NormalTok{(p1}\SpecialCharTok{*}\NormalTok{n1,p2}\SpecialCharTok{*}\NormalTok{n2)}
\NormalTok{n }\OtherTok{\textless{}{-}} \FunctionTok{c}\NormalTok{(n1,n2)}
\FunctionTok{prop.test}\NormalTok{( success, n, }\AttributeTok{alternative=}\StringTok{"two.sided"}\NormalTok{, }\AttributeTok{correct=}\ConstantTok{FALSE}\NormalTok{)}
\end{Highlighting}
\end{Shaded}

\begin{verbatim}
## 
##  2-sample test for equality of proportions without continuity correction
## 
## data:  success out of n
## X-squared = 38.962, df = 1, p-value = 4.321e-10
## alternative hypothesis: two.sided
## 95 percent confidence interval:
##  -0.05356945 -0.02891741
## sample estimates:
##     prop 1     prop 2 
## 0.02419355 0.06543698
\end{verbatim}

\hypertarget{interpretaciuxf3n-del-test-3}{%
\subsection{Interpretación del
test}\label{interpretaciuxf3n-del-test-3}}

El valor p es 4.321e-10 y, dado que es menor que α=0.05, No estamos en
la zona de aceptación de la hipótesis nula. Por tanto, no podemos
afirmar que las diferencias de proporciones sean significativamente
diferentes con un nivel de confianza del 95\%.

\hypertarget{resumen-y-conclusiones}{%
\section{Resumen y conclusiones}\label{resumen-y-conclusiones}}

\begin{longtable}[]{@{}
  >{\centering\arraybackslash}p{(\columnwidth - 6\tabcolsep) * \real{0.1200}}
  >{\raggedright\arraybackslash}p{(\columnwidth - 6\tabcolsep) * \real{0.2000}}
  >{\raggedright\arraybackslash}p{(\columnwidth - 6\tabcolsep) * \real{0.2400}}
  >{\raggedright\arraybackslash}p{(\columnwidth - 6\tabcolsep) * \real{0.4400}}@{}}
\toprule()
\begin{minipage}[b]{\linewidth}\centering
N
\end{minipage} & \begin{minipage}[b]{\linewidth}\raggedright
PREGUNTA
\end{minipage} & \begin{minipage}[b]{\linewidth}\raggedright
RESULTADO
\end{minipage} & \begin{minipage}[b]{\linewidth}\raggedright
CONCLUSIÓN
\end{minipage} \\
\midrule()
\endhead
1.1 & ¿Cuál es el intervalo de confianza de la nota entre los
estudiantes al 90\%? & El Intervalo confianza a un 90\% esta 2.637309 y
2.670953 & El valor medio de confianza a un 90\% es de 2.654131 \\
1.2 & ¿Cuál es el intervalo de confianza de la nota entre los
estudiantes al 95\%? & El Intervalo confianza a un 95\% esta 2.634086 y
2.674176 & El valor medio de confianza a un 95\% es de 2.654131 \\
2 & ¿Ser atleta influye en la nota? & Valor Crítico al 95\% = 1.960538;
P-Value = 3.689891e-09; Valor Observado = -5.910309 & Si, ser atleta
influye en la nota \\
3.1 & ¿Las mujeres obtienen mejor nota que los hombres a un nivel de
confianza del 95\%? & Valor Crítico al 95\% = 1.960545; P-Value
=1.704394e-12; Valor Observado = 7.078735 & En un nivel de confianza del
95\% podemos afirmar que los hombres tienen mejor nota que las
mujeres \\
3.2 & ¿Las mujeres obtienen mejor nota que los hombres a un nivel de
confianza del 90\%? & Valor Crítico al 90\% = 1.645227; P-Value =
1.704394e-12; Valor Observado = 7.078735 & En un nivel de confianza del
90\% podemos afirmar que los hombres tienen mejor nota que las
mujeres \\
4 & ¿Hay diferencias significativas en la nota según la raza? & Valor
Crítico al 95\% = 1.960549; P-Value =1.99014e-21; Valor Observado
=9.559319 & En un nivel de confianza del 95\% podemos afirmar que los
estudiantes Blancos tienen mejor nota que los estudiantes Negros \\
5 & ¿La proporción de atletas en la población es inferior al 5\%? &
Valor Crítico al 95\% = 1.644854; P-Value = 1; Valor Observado =
0.04689388 & En un nivel de confianza del 95\% se afirma que la
proporción de atletas es menor al 5\% \\
6 & ¿Hay más atletas entre los hombres que entre las mujeres? & Valor
Crítico al 95\% = -1.959964; P-Value = 4.321099e-10; Valor Observado =
-6.241964 & En un nivel de confianza del 95\% se afirma que la
proporción de atletas hombres es mayor que la proporción de atletas
mujeres \\
\bottomrule()
\end{longtable}

\hypertarget{resumen-ejecutivo}{%
\section{Resumen ejecutivo}\label{resumen-ejecutivo}}

Para realizar el resumen ejecutivo se va a responder a todas las
cuestiones planteadas en el enunciado:

\begin{itemize}
\item
  \textbf{P1.1 ¿Cuál es el intervalo de confianza de la nota entre los
  estudiantes?} El valor medio de confianza a un 90\% es de 2.654131,
  siendo el limite inferior 2.637309 y el superior 2.670953.
\item
  \textbf{P1.2 ¿Cuál es el intervalo de confianza de la nota entre los
  estudiantes?} El valor medio de confianza a un 95\% es de 2.654131,
  siendo el limite inferior 2.634086 y el superior 2.674176
\item
  \textbf{P2. ¿Ser atleta influye en la nota?} Si, a un nivel de
  confianza del 95\% podemos concluir que las personas no atletas tienen
  mejor nota de las personas que si lo son.
\item
  \textbf{P3.1. ¿Las mujeres obtienen mejor nota que los hombres a un
  nivel de confianza del 95\%?} Si, a un nivel de confianza del 95\%
  podemos afirmar que los hombres tienen mejor nota que las mujeres.
\item
  \textbf{P3.2. ¿Las mujeres obtienen mejor nota que los hombres a un
  nivel de confianza del 90\%?} Si, a un nivel de confianza del 90\%
  podemos afirmar que los hombres tienen mejor nota que las mujeres.
\item
  \textbf{P4. ¿Hay diferencias significativas en la nota según la raza?}
  Si, a un nivel de confianza del 95\% podemos afirmar que los
  estudiantes Blancos tienen mejor nota que los estudiantes Negros.
\item
  \textbf{P5. ¿La proporción de atletas en la población es inferior al
  5\%?} Si, a un nivel de confianza del 95\% se afirma que la proporción
  de atletas es menor al 5\%
\item
  \textbf{P6. ¿Hay más atletas entre los hombres que entre las mujeres?}
  Si, a un nivel de confianza del 95\% se afirma que la proporción de
  atletas hombres es mayor que la proporción de atletas mujeres.
\end{itemize}

\end{document}
